\section{Background}
    \subsection{Turkish agent nominalizers}
    Turkish derivational suffixes \textit{-CI} and \textit{-(y)IcI} are two productive agent nominalizers that take noun and verb bases respectively as illustrated in (\ref{nominalizers}).
    
    \begin{exe}
    \ex \label{nominalizers}
    \begin{multicols}{2}
    \begin{xlist}
        \ex \gll sol-cu \\ left-CI \\
        \glt `leftist'
       \columnbreak
        \ex \gll koş-ucu \\ run-(y)IcI \\
        \glt 'runner'
    \end{xlist}
    \end{multicols}
    \end{exe}
 Categories for the derivational suffix \textit{-CI} is outlined in \cite{goksel2004turkish} with uses in denoting profession, ideological adherence, activity engagement, habitual involvement, and personal preference. The derivations of \textit{-CI} and \textit{-(y)IcI} can be represented in construction morphology terms \citep{booij2005compounding} like in (\ref{derivations}).
 
 \begin{exe}
    \ex \label{derivations}
    \begin{xlist}
        \ex $[[$X$]_{N}$ -CI$]_{Y_{i}}$ is a derivation where Y is a Noun with SEM$_{i}$ to a Noun X.
        
        \ex $[[$X$]_{V}$ -(y)IcI$]_{Y}$ is a derivation where Y is a Noun which Vs
    \end{xlist}
 \end{exe}
The SEM$_{i}$ for the derivations are not constant, it is dependent on the noun (\ref{dependent}). The noun base is not interpreted as a definite noun, it is read as a non-referential noun.

\begin{exe}
    \ex \label{dependent}
    \begin{xlist}
    \ex \gll deniz-ci \\ sea-{\Ci} \\ \glt `sailor, is someone$_{Y}$ who sails$_{i}$ the seas$_{X}$.'
    \ex \gll kitap-çı \\ book-{\Ci} \\ \glt `bookseller, is someone/a place$_{Y}$ that sells$_{i}$ books$_{X}$.'
    \end{xlist}
\end{exe}
    
    \rule{\columnwidth}{0.2pt}
    
    \subsection{Sakha agent nominalizer}
    \cite{baker2009agent} propose the semantic denotation in (\ref{sakhaagent}) for the agent nominalizer \textit{-AAccI} in Sakha (a Turkic language spoken in northeastern Russia).
    
    \begin{exe}
    \ex \label{sakhaagent} $[[$-AAccI$]]_{\textless vt, et \textgreater}$ = $\lambda{P_{\textless vt \textgreater}}.^{\frown}\lambda{x}. Gen\,e \; \,P(e) \wedge AG(e,x)$ 
    \end{exe}
    
    This denotation is also what \cite{baker2009agent} proposes for the English \textit{-er}. `$^{\frown}$' is an operator proposed by \cite{chierchia1985formal} that takes predicative expressions and returns singular ones. 
    \rule{\columnwidth}{0.2pt}
    
    \subsection{Semantic machinery}
    
    In my analysis I use Restrict \citep{chung2003restriction}, and Neo-Davidsonian \citep{kratzer1996severing,maienborn2011event, parsons1990events} approach. Restrict is not an operation of saturation. Below I give an abstract representation
    
    \begin{framed}
    $\alpha$ that has daughters $\beta$ and $\gamma$ \\
    If $x \in{dom([[\gamma]])}$ and $[[\beta]]$ is a set of $x$ \\
    Then $[[\alpha]] = [[\beta]] \wedge [[\gamma]]$
    \end{framed}
    
    In Neo-Davidsonian approach the predications are evaluated as an event, where all the arguments hold relations to the event. Following from \cite{kratzer1996severing}, I adopt the claim that all verbs are of D$_{\textless e,vt\textgreater}$ where D$_v$ represents the domain of events.