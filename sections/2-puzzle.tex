\section{Puzzle}
    \subsection{Meaning shifts}
    While the agent nominalizer \textit{-(y)IcI} does not change derivational meaning in any context, \textit{-CI} allows for meaning shifts depending on the context as shown in (\ref{contextchange}).
    
    \begin{exe}
    \ex \label{contextchange}
    \begin{xlisti}
    \exi{\textit{-CI}} 
    \ex \begin{xlista}
    \ex \label{speakerA} \gll \textit{Ben}$_i$ \textit{dergi} \textit{oku-ma-yı} \textit{sev-iyor-um.} \\ {\First}.{\Sg}[{\Nom}] magazine read-{\Nmlz}-{\Acc} like-{\Prog}-{\First}.{\Sg} \\
    \glt `I like to read magazines.'

    \ex \label{speakerB} \gll \textit{Ben}$_j$ \textit{kitap-çı-yım} \\ {\First}.{\Sg}[{\Nom}] book-{\Ci}-{\Cop}.{\First}.{\Sg} \\
    \glt Literal: `I am book-{\Ci}.'
    Meaning: `I am a book-lover.'
    \end{xlista}
    
    \exi{\textit{-(y)IcI}} 
    \ex \begin{xlista}
    \ex \gll Ben$_i$ kitap yaz-may-ı sev-iyor-um. \\ {\Fsg}[{\Nom}] book write-{\Nmlz}-{\Acc} like-{\Prog}-{\Fsg} \\
    \glt `I like to write books.' 
    
    \ex \gll Ben$_j$ oku-yucu-yum. \\ {\Fsg} read-yIcI-{\Cop}.{\Fsg} \\
    \glt `*I like to read books.' \\ `I read books.' 
    
    \end{xlista}
    \end{xlisti}
    \end{exe}
The context dependency or changeability of \textit{-CI} dereivations is a problem for strong lexicalist view that derivations are opaque to contextual or syntactic operations \citep{bresnan1995lexical}.
 
    \rule{\columnwidth}{0.4pt}
    
    
    \subsection{Compositionality}
    
    \cite{baker2009agent} investigate the possible modifiers of \textit{-AAccI} bases. They report that agent oriented adverbs are unable to modify the verb base for the \textit{-AAccI}. Their observations hold the same for Turkish \textit{-(y)IcI} too (\ref{modification}).
    
    \begin{exe}
    \ex \label{modification}
    \begin{xlist}
    \ex \gll Sev-erek oku-mak güzel-dir. \\ like-{\Cvb} read-{\Nmlz} nice-{\Cop}[{\Tsg}] \\
    \glt `Studying by heart is nice.'
    
    \ex \gll *Sev-erek oku-yucu güzel-dir. \\ like-{\Cvb} read-yIcI nice-{\Cop}[{\Tsg}] \\
    \end{xlist}
    \end{exe}
    
    While the base for \textit{-(y)IcI} derivation can not be modified by an adverb, the base for the \textit{-CI} derivation can be modified (\ref{modification2}). However when the modifier is compatible with both the derived noun and the base, the prominent reading is not the modification of the base (\ref{mod2a}). If the modifier is incompatible with the derived noun then the modification of the base is the only reading (\ref{mod2b}).
    
    
    \begin{exe}
    \ex \label{modification2}
\begin{multicols}{2}  
    \begin{xlist}
    \ex \label{mod2a} \gll eski vazo-cu \\ old vase-{\Ci} \\ 
    \glt `a vase-seller who is old' \\ `a seller of old vases'

    \ex \label{mod2b} \gll antika vazo-cu \\ antique vase-{\Ci} \\ 
    \glt`*a vase-seller who is antique' \\ `a seller of antique vases'
    

    \end{xlist}
\end{multicols}
    \end{exe}
